\documentclass[AutoFakeBold,AutoFakeSlant]{beamer}
\usepackage{ctex, hyperref}
\usepackage[T1]{fontenc}

% other packages
\usepackage{latexsym,amsmath,xcolor,multicol,booktabs,calligra}
\usepackage{graphicx,pstricks,listings,stackengine}

\author{@六边形游乐场}
\title{Beamer 制作演示}
\subtitle{一个简单的例子}
\institute{六边形宇宙驻B站办事处}
\date{2022年12月2日}
\usepackage{Tsinghua}

% defs
\def\cmd#1{\texttt{\color{red}\footnotesize $\backslash$#1}}
\def\env#1{\texttt{\color{blue}\footnotesize #1}}
\definecolor{deepblue}{rgb}{0,0,0.5}
\definecolor{deepred}{rgb}{0.6,0,0}
\definecolor{deepgreen}{rgb}{0,0.5,0}
\definecolor{halfgray}{gray}{0.55}

\lstset{
    basicstyle=\ttfamily\small,
    keywordstyle=\bfseries\color{deepblue},
    emphstyle=\ttfamily\color{deepred},    % Custom highlighting style
    stringstyle=\color{deepgreen},
    numbers=left,
    numberstyle=\small\color{halfgray},
    rulesepcolor=\color{red!20!green!20!blue!20},
    frame=shadowbox,
}


\begin{document}

\kaishu
\begin{frame}
    \titlepage
    \begin{figure}[htpb]
        \begin{center}
            \includegraphics[width=0.2\linewidth]{pic/Tsinghua_University_Logo.eps}
        \end{center}
    \end{figure}
\end{frame}

\begin{frame}
\tableofcontents[sectionstyle=show,subsectionstyle=show/shaded/hide,subsubsectionstyle=show/shaded/hide]
\end{frame}


\section{基本框架}
\begin{frame}[fragile]{页面结构}

1. 页面\\
每一页都在 frame 环境里,没有例外\\
\vspace{1em}
2. 封面\\
封面也是,只是不同的模版会有一些各自的设计\\
\vspace{1em}
3. 目录\\
一般会在目录里显示章和节,和论文模版类似,可以控制显示到哪一级和具体的样式
\begin{itemize}
    \item 章 \verb|\section{...}|
    \item 节 \verb|\subsection{...}|
\end{itemize}
\vspace{1em}
从上面几句可以看出,只要直接在 frame 里打字就可以直接形成一页 PPT 的内容,是不是很简单\\

\end{frame}

\section{进阶使用}
\subsection{页面小知识}

\begin{frame}[fragile]{标题:标题可以这样写}
    页面标题可以直接写在 frame 环境的后面,就像本页
\end{frame}

\begin{frame}[fragile]
\frametitle{标题:也可以使用专门的命令}
    也可以使用这个更加通用的命令\\
    \vspace{1em}
\begin{verbatim}
    \frametitle{...}
\end{verbatim}
    \vspace{1em}
    大部分模版也都支持\\
\end{frame}

\begin{frame}[fragile]{强调:文本突出}
\begin{itemize}
    \item \alert{强调} \verb|\alert{...}|
    \item \colorbox{yellow}{底色} \verb|\colorbox{...}|
    \item \underline{下划线} \verb|\underline{...}|
    \item \textbf{Bold} \verb|\textbf{...}|
    \item \emph{Italic} \verb|\emph{...}|
    \item 关于这个模版的\textbf{中文加粗} ,请看注释
% 这个模版的中文加粗有点问题,调整它需要介绍过多内容
% 关于LaTeX 字体的内容,可以参考学习 https://zhuanlan.zhihu.com/p/538459335
% 这个模版里中文加粗的问题,我的处理方式比较简单,直接将第一行修改为 \documentclass[AutoFakeBold,AutoFakeSlant]{beamer} 增加了 AutoFakeBold,效果一般,但是能用了
\end{itemize}


\end{frame}

\begin{frame}{强调:Block 环境}
\begin{block}{小知识1}
    色块可以让段落更加鲜明
\end{block}
\vspace{1em}
\begin{block}{小知识2}
    阅读体验更好
\end{block}
\end{frame}

\begin{frame}[fragile]{动画}
这毕竟是PDF,不要指望酷炫的动画效果,只能实现控制每页出现一点的渐变显示效果
\vspace{1em}
    \begin{itemize}
        \item 使用 \verb|\pause| 命令
        \item 使用 \verb|\onslde<n>{...}| 命令
        \item 使用 \verb|\only<n>{...}| 命令
    \end{itemize}
\end{frame}

\begin{frame}{动画:Pause 命令举例}
首先,用 \LaTeX 做 PPT 也挺好\\
\vspace{1em}
\pause
其次,用 Keynote 画 PPT 也很好,改天单独做几期视频 \\
\end{frame}



\subsection{其他 \LaTeX 常见元素}
\begin{frame}{常见元素}
比如说:
    \begin{itemize}
        \item 有序列表、无序列表: itemize、enumerate 环境
        \item 插入图片: figure 环境
        \item 插入表格: table 环境
        \item 插入公式: equation 环境
        \item 参考文献的引用: ref、footnote等命令
    \end{itemize}
    \vspace{1em}
    \begin{block}{小提示}
    以上这些内容的插入,跟前几期视频介绍的,在论文中插入的语法一摸一样,请回看前几期视频,在此不做赘述 \\ 
    \end{block}
\end{frame}
\end{document}