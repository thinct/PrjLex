%导言区
\documentclass{ctexart}  %ctexbook ctexrep

% \newcommand 定义命令
% 命令只能由字母组成,不能以\end开头
%[]内为可选参数
% \newcommand<命令名称>[<参数个数>][<首参数默认值>]{<具体定义>}

%\newcommand可以是简单字符串替换,例如:
%\emph{text}表示强调
%使用\PRC 相当于 People's Republic of \emph{China} 这一串内容 
\newcommand\PRC{People's Republic of \emph{China}}

%\newcommand也可以使用参数
%参数个数可以从1到9,使用时用#1,#2,...,#9表示
\newcommand\love[2]{#1 喜欢 #2}
\newcommand{\hateby}[2]{#2 不受 #1 喜欢}

%\newcommand的参数也可以有默认值
%指定参数个数的同时指定首个参数的默认值 
%那么这个命令的第一个参数就成为了可选参数(调用时要使用[]指定 不然就使用默认值)
%下述命令有三个参数 其中第一个参数#1为可选参数 默认值为喜欢
\newcommand{\loves}[3][喜欢]{#2#1#3}

%\renewcommand重定义命令
%与\newcommand命令作用和用法相同,但只能用于已有命令
%\renewcommand<命令名称>[<参数个数>][<首参数默认值>]{<具体定义>}
\renewcommand{\abstractname}{简介} %重新定义\abstractname

%定义和重定义环境
%[]内可选 {}必填
%\newenvironment{<环境名称>}[<参数个数>][<首参数默认值>]{<环境前定义>}
%							{<环境后定义>}
%\renewenvironment{<环境名称>}[<参数个数>][<首参数默认值>]{<环境前定义>}
%							{<环境后定义>}

%为book类中定义摘要(abstract)环境
\newenvironment{myabstract}[1][摘要]
{\small 
	\begin{center} \bfseries #1 \end{center}
	\begin{quotation}}
	{\end{quotation}}   

%环境参数只有<环境前定义>中可以使用的参数
%<环境后定义>中不能再使用环境参数
%如果需要,可以先把前面得到的参数保存在一个命令中,在后面使用:

\newenvironment{Quotation}[1]
{\newcommand{\quotesource}{#1}
	\begin{quotation}}
	{\par\hfill---《\textit{\quotesource}》
\end{quotation}}

%正文区
\begin{document}
	\PRC
	
	\love{猫}{鱼}
	
	\hateby{猫}{萝卜}
	
	\loves{猫}{鱼} %不指定可选参数的取值 就使用默认值
	
	\loves[最爱]{猫}{鱼} %[]指定可选参数的取值 此时将覆盖默认值
	
	\begin{abstract}%\abstractname命令在abstract环境中自动调用
		这是一段摘要。。。
	\end{abstract}
	
	\begin{myabstract} %使用自定义环境 使用默认参数值
		这是一段自定义格式的摘要。。。
	\end{myabstract}
	
	\begin{myabstract}[我的摘要] %使用自定义环境 传入参数值 覆盖默认参数
		这是一段自定义格式的摘要。。。
	\end{myabstract}
	
	\begin{Quotation}{易$\cdot$乾}
		初九,潜龙勿用。
	\end{Quotation}
	
	定义命令和环境是进行\LaTeX{}格式定制、达成内容和格式分离目标的利器。使用自定义的命令和环境把字体、字号、缩进、对齐、间距等各种琐细的内容包装起来,辅以一个有意义的名字,可以使文档结构清晰、代码整洁、易于维护。在使用宏定义的功能时,要综合利用各种已有命令、环境、变量等功能,事实上,前面所介绍的长度变量与盒子、字体字号等内容,大多不直接出现在文档正文中,而主要都是用在实现各种结构化的宏定义中。
	
\end{document}