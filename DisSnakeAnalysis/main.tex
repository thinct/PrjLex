\documentclass[AutoFakeBold,AutoFakeSlant]{beamer}
\usetheme{metropolis}           % Use metropolis theme
\setbeamercovered{transparent}
\usepackage{listings}

% 支持中文的设置
\usepackage{ctex, hyperref}
\usepackage[T1]{fontenc}
% other packages
\usepackage{latexsym,amsmath,xcolor,multicol,booktabs,calligra}
\usepackage{graphicx,pstricks,listings,stackengine}


\title{\textbf{贪吃蛇}\\反汇编代码分析报告}
\date{\today}
\author{THINCT}
\begin{document}
	\maketitle 
	
	
	\section{SnakeGame::update}
	
	\subsection{函数前置初始化}
	
	% Using typewriter font: \ttfamily inside \lstset
	\begin{frame}[fragile]
		\frametitle{EBX 代替当前的函数栈底}
		\begin{lstlisting}[language={[x86masm]Assembler}]
.text:004079C0  push    ebx
.text:004079C1  mov     ebx, esp
.text:004079C3  sub     esp, 8
.text:004079C6  and     esp, -8
.text:004079C9  add     esp, 4
.text:004079CC  push    ebp
.text:004079CD  mov     ebp, [ebx+4]
.text:004079D0  mov     [esp+4], ebp
.text:004079D4  mov     ebp, esp
		\end{lstlisting}
		\begin{enumerate}
			\item 当eip在.text:004079C0处,esp所指向的是ret addr. 
			\item 当eip在.text:004079C1处,ebx 所指向的是esp-4.此时:
			\begin{itemize}
					\item ebx+4指向的是ret addr
					\item ebx+8 指向的是第一个参数
			\end{itemize}
		\end{enumerate}
	\end{frame}
	
	
	\begin{frame}[fragile]{EBX 代替当前的函数栈底}
		\begin{lstlisting} [language={[x86masm]Assembler}]
.text:004079C3  sub     esp, 8
.text:004079C6  and     esp, 0FFFFFFF8h
.text:004079C9  add     esp, 4
.text:004079CC  push    ebp
		\end{lstlisting} 
		esp实现了向下最近的8的倍数取证。比如12取整就是8,16取整就是16,18取整就是16.因为是针对栈结构地址取整,所以越是往小的方向越安全,因为对于栈结构来讲,越小的地址是没有用过的地址。所以后面的ebp,esp, ebp只能作为局部变量的索引,而对于参数的索引,用ebx比较合适。
		
		\textbf {总结:}\\
		\emph {对于这个函数来讲,并不是按照套路ebp作为局部变量和函数参数的唯一参考.} 
	\end{frame}
	
\end{document}
