% !TeX program = xelatex 

\documentclass[AutoFakeBold,AutoFakeSlant]{beamer}
\usetheme{metropolis}           % Use metropolis theme
\setbeamercovered{transparent}
\usepackage{listings}
\usepackage[dvipsnames]{xcolor}

% 支持中文的设置
\usepackage{xeCJK}
\usepackage{fontspec}
\setCJKmainfont[ItalicFont=思源宋体,BoldFont=SourceHanSerifSC-Bold]{Source Han Serif SC}
\newcommand{\KaiTi}{\CJKfontspec{楷体}}%用命令\fzkaiti调用方正楷体简体

% other packages
\usepackage{latexsym,amsmath,xcolor,multicol,booktabs,calligra}
\usepackage{graphicx,pstricks,listings,stackengine}

\lstnewenvironment{x86asmcode}[1][]%
{
	\lstset{
		tabsize=4,
		breaklines=true,
		breakatwhitespace=true, % 在空格处断行
		language={[x86masm]Assembler},
		escapeinside=``,
		basicstyle=\ttfamily,
		keywordstyle=\bfseries\color{NavyBlue},
		commentstyle=\itshape\color{black!50!white},
		literate={\ \ }{{\ \ \ \ }}4,
		#1
	}
}%
{}

\title{\textbf{贪吃蛇}\\反汇编代码分析报告}
\date{\today}
\author{THINCT}
\begin{document}
	\maketitle 
	
	
	\section{SnakeGame::update}
	
	\subsection{函数前置初始化}
	
	% Using typewriter font: \ttfamily inside \lstset
	\begin{frame}[fragile]
		\frametitle{EBX 代替当前的函数栈底}
		\begin{x86asmcode}
004079C0  push  ebx
004079C1  mov   ebx, esp
004079C3  sub   esp, 8
004079C6  and   esp, -8
004079C9  add   esp, 4
004079CC  push  ebp
004079CD  mov   ebp, [ebx+4]
004079D0  mov   [esp+4], ebp
004079D4  mov   ebp, esp\end{x86asmcode}
		\begin{enumerate}
			\item 当eip在.text:004079C0处,esp所指向的是ret addr. 
			\item 当eip在.text:004079C1处,ebx 所指向的是esp-4.此时:
			\begin{itemize}
					\item ebx+4指向的是ret addr
					\item ebx+8 指向的是第一个参数
			\end{itemize}
		\end{enumerate}
	\end{frame}
	
	
	\begin{frame}[fragile]{EBX 代替当前的函数栈底}
        \begin{x86asmcode}
004079C3  sub   esp, 8
004079C6  and   esp, 0FFFFFFF8h 
004079C9  add   esp, 4
004079CC  push  ebp\end{x86asmcode} 
		esp实现了向下最近的8的倍数取证。比如12取整就是8,16取整就是16,18取整就是16.因为是针对栈结构地址取整,所以越是往小的方向越安全,因为对于栈结构来讲,越小的地址是没有用过的地址。所以后面的ebp,esp, ebp只能作为局部变量的索引,而对于参数的索引,用ebx比较合适。
		
		\textbf {总结:}\\
		\emph {对于这个函数来讲,并不是按照套路ebp作为局部变量和函数参数的唯一参考.} 
	\end{frame}
	
	\begin{frame}[fragile]
		\frametitle{operator += 传参}
		\begin{x86asmcode}
004079FF  mov   eax, [ebx+8]
00407A02  mov   ecx, [eax+4]
00407A05  push  ecx
00407A06  mov   edx, [eax]
00407A08  push  edx
00407A09  mov   eax, [ebp-2Ch]
00407A0C  add   eax, 28h ; '('
00407A0F  push  eax
00407A10  call  sf::operator+=(sf::Time &,sf::Time)\end{x86asmcode}
		\begin{itemize}
		\item 从0x00407A09到0x00407A0F是第一个参数,已知[ebp-2Ch]为this,所以第一个参数为this->offset28h,并且为sf::Time引用类型.所以\textbf{\textit{sf::Time* this->offset28h}}.
	\end{itemize}
	\end{frame}
	
	\begin{frame}[fragile]{operator += 传参}
		\begin{x86asmcode}
004079FF  mov   eax, [ebx+8]
00407A02  mov   ecx, [eax+4]
00407A05  push  ecx
00407A06  mov   edx, [eax]
00407A08  push  edx
00407A09  mov   eax, [ebp-2Ch]
00407A0C  add   eax, 28h ; '('
00407A0F  push  eax
00407A10  call  sf::operator+=(sf::Time &,sf::Time)\end{x86asmcode}
		\begin{itemize}
			\item 0x004079FF已推导出为当前函数的第一个参数,而0x00407A02到0x00407A08是连续的内存,从call得知这个连续的内存是sf::Time类型,所以推导出[ebx+8]是sf::Time*类型,即\textit{\textbf{sf::Time* [ebx+8]}}
		\end{itemize}
	\end{frame}
	
	\begin{frame}[fragile]{operator += 传参}
		\textit{\textbf{总结:}} \\
		operator += 第一个参数是传地址,第二个参数是传值,只不过sf::Time的内存是8个字节,所以从起始地址连续压栈2次.本重载函数主要需要掌握的是:\textit{\textbf{不能根据push来判断函数的参数个数}}.
	\end{frame}
	
	\begin{frame}[fragile]
		\frametitle{函数的返回值才是第一个参数}
		\begin{x86asmcode}
00407A19 mov    ecx, [ebp-2Ch]
00407A1C movss  xmm0, ds:__real@3f800000
00407A24 divss  xmm0, dword ptr [ecx+24h]
00407A29 push   ecx
00407A2A movss  dword ptr [esp], xmm0
00407A2F lea    edx, [ebp-28h]
00407A32 push   edx
00407A33 call   ds:__imp_?seconds@sf@@YA?AVTime@1@M@Z ; sf::seconds(float) \end{x86asmcode}
		0x00407A2F处压栈的是第一个参数,为局部变量(\textbf{暂存临时返回值}),0x00407A29和0x00407A2A压栈第二个参数,其中push只是占位作用,0x00407A2A才是第二个参数的值,也就是计算出来的浮点数.
	\end{frame}
	
	\begin{frame}[fragile]{函数的返回值才是第一个参数}
		\begin{x86asmcode}
7AC94C90  movss  xmm0,dword ptr [esp+8]  
7AC94C96  mulss  xmm0,dword ptr ds:[7AC982B8h]  
7AC94C9E  call   7AC9600E  
7AC94CA3  mov    ecx,dword ptr [esp+4]  
7AC94CA7  mov    dword ptr [ecx],eax  
7AC94CA9  mov    eax,ecx  
7AC94CAB  mov    dword ptr [ecx+4],edx  
7AC94CAE  ret \end{x86asmcode}
		观察调用的函数,分别从0x7AC94CA3和0x7AC94CA9可知:第一个参数也是该函数的返回值,所以可以推断出:\textbf{函数的返回值才是第一个参数,并且该函数其实只有一个参数},即0x00407A2A处压栈的参数.
	\end{frame}
	
	
	\begin{frame}[fragile]
		\frametitle{参数与函数可能隔了几个call}
		\begin{x86asmcode}
00407B1A  push    1 ; includesHead
00407B1C  mov     ecx, [ebp-2Ch]
00407B1F  add     ecx, 1408h
00407B25  call    sf::Transformable::getPosition(void)
00407B2B  push    eax ; location
00407B2C  mov     ecx, [ebp-2Ch]
00407B2F  add     ecx, 30h ; this
00407B32  call    Snake::collidesWith(sf::Vector2<float> const &,bool)\end{x86asmcode}
		0x00407B1A处压栈参数,在0x00407B25 call之后并没有平栈.通过动态调试得知\textbf{0x00407B25前后esp没有变化,说明该call是没有参数的},跟进0x00407B25 call直接将以地址给eax了,直接说明了\textbf{0x00407B1A push不是0x00407B25 call使用的}.	
	\end{frame}
	
		\begin{frame}[fragile]{参数与函数可能隔了几个call}
		\begin{x86asmcode}
00407B1A  push    1 ; includesHead
00407B1C  mov     ecx, [ebp-2Ch]
00407B1F  add     ecx, 1408h
00407B25  call    sf::Transformable::getPosition(void)
00407B2B  push    eax ; location
00407B2C  mov     ecx, [ebp-2Ch]
00407B2F  add     ecx, 30h ; this
00407B32  call    Snake::collidesWith(sf::Vector2<float> const &,bool)\end{x86asmcode}
		分析0x00407B32 call,得出0x00407B2B是第一个参数,上面的0x00407B1A是第二个参数.	是因为动态调试发现经过0x00407B32 call前后,esp变化为8,所以直接找最近的栈的两个push即可.
	\end{frame}
	
	\begin{frame}[fragile]{参数与函数可能隔了几个call}
		\textbf{总结:}
		\begin{enumerate}
			\item 通过动态调试观察esp变化来判断参数的个数
			\item 经过编译器的优化,函数的push可能在其他call之前进行压栈的
		\end{enumerate}
	\end{frame}
	
\end{document}
