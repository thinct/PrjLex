% !TeX program = xelatex 

\documentclass[AutoFakeBold,AutoFakeSlant]{beamer}
\usetheme{metropolis}           % Use metropolis theme
\setbeamercovered{transparent}
\usepackage{listings}
\usepackage[dvipsnames]{xcolor}

% 支持中文的设置
\usepackage{xeCJK}
\usepackage{fontspec}
\setCJKmainfont[ItalicFont=思源宋体,BoldFont=SourceHanSerifSC-Bold]{Source Han Serif SC}
\newcommand{\KaiTi}{\CJKfontspec{楷体}}%用命令\fzkaiti调用方正楷体简体

% other packages
\usepackage{latexsym,amsmath,xcolor,multicol,booktabs,calligra}
\usepackage{graphicx,pstricks,listings,stackengine}

\lstnewenvironment{x86asmcode}[1][]%
{
	\lstset{
		tabsize=4,
		breaklines=true,
		breakatwhitespace=true, % 在空格处断行
		language={[x86masm]Assembler},
		escapeinside=``,
		basicstyle=\ttfamily,
		keywordstyle=\bfseries\color{NavyBlue},
		commentstyle=\itshape\color{black!50!white},
		literate={\ \ }{{\ \ \ \ }}4,
		#1
	}
}%
{}

\title{\textbf{贪吃蛇}\\反汇编代码分析报告}
\date{\today}
\author{THINCT}
\begin{document}
	\maketitle 
	
	
	\section{SnakeGame::update}
	
	\subsection{函数前置初始化}
	
	% Using typewriter font: \ttfamily inside \lstset
	\begin{frame}[fragile]
		\frametitle{EBX 代替当前的函数栈底}
		\begin{x86asmcode}
004079C0  push  ebx
004079C1  mov   ebx, esp
004079C3  sub   esp, 8
004079C6  and   esp, -8
004079C9  add   esp, 4
004079CC  push  ebp
004079CD  mov   ebp, [ebx+4]
004079D0  mov   [esp+4], ebp
004079D4  mov   ebp, esp\end{x86asmcode}
		\begin{enumerate}
			\item 当eip在.text:004079C0处,esp所指向的是ret addr. 
			\item 当eip在.text:004079C1处,ebx 所指向的是esp-4.此时:
			\begin{itemize}
					\item ebx+4指向的是ret addr
					\item ebx+8 指向的是第一个参数
			\end{itemize}
		\end{enumerate}
	\end{frame}
	
	
	\begin{frame}[fragile]{EBX 代替当前的函数栈底}
        \begin{x86asmcode}
004079C3  sub   esp, 8
004079C6  and   esp, 0FFFFFFF8h 
004079C9  add   esp, 4
004079CC  push  ebp\end{x86asmcode} 
		esp实现了向下最近的8的倍数取证。比如12取整就是8,16取整就是16,18取整就是16.因为是针对栈结构地址取整,所以越是往小的方向越安全,因为对于栈结构来讲,越小的地址是没有用过的地址。所以后面的ebp,esp, ebp只能作为局部变量的索引,而对于参数的索引,用ebx比较合适。
		
		\textbf {总结:}\\
		\emph {对于这个函数来讲,并不是按照套路ebp作为局部变量和函数参数的唯一参考.} 
	\end{frame}
	
	\begin{frame}[fragile]
		\frametitle{operator += 传参}
		\begin{x86asmcode}
004079FF  mov   eax, [ebx+8]
00407A02  mov   ecx, [eax+4]
00407A05  push  ecx
00407A06  mov   edx, [eax]
00407A08  push  edx
00407A09  mov   eax, [ebp-2Ch]
00407A0C  add   eax, 28h ; '('
00407A0F  push  eax
00407A10  call  sf::operator+=(sf::Time &,sf::Time)\end{x86asmcode}
		\begin{itemize}
		\item 从0x00407A09到0x00407A0F是第一个参数,已知[ebp-2Ch]为this,所以第一个参数为this->offset28h,并且为sf::Time引用类型.所以\textbf{\textit{sf::Time* this->offset28h}}.
	\end{itemize}
	\end{frame}
	
	\begin{frame}[fragile]{operator += 传参}
		\begin{x86asmcode}
004079FF  mov   eax, [ebx+8]
00407A02  mov   ecx, [eax+4]
00407A05  push  ecx
00407A06  mov   edx, [eax]
00407A08  push  edx
00407A09  mov   eax, [ebp-2Ch]
00407A0C  add   eax, 28h ; '('
00407A0F  push  eax
00407A10  call  sf::operator+=(sf::Time &,sf::Time)\end{x86asmcode}
		\begin{itemize}
			\item 0x004079FF已推导出为当前函数的第一个参数,而0x00407A02到0x00407A08是连续的内存,从call得知这个连续的内存是sf::Time类型,所以推导出[ebx+8]是sf::Time*类型,即\textit{\textbf{sf::Time* [ebx+8]}}
		\end{itemize}
	\end{frame}
	
	\begin{frame}[fragile]{operator += 传参}
		\textit{\textbf{总结:}} \\
		operator += 第一个参数是传地址,第二个参数是传值,只不过sf::Time的内存是8个字节,所以从起始地址连续压栈2次.本重载函数主要需要掌握的是:\textit{\textbf{不能根据push来判断函数的参数个数}}.
	\end{frame}
	
\end{document}
