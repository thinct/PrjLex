% !TeX program = xelatex 

\documentclass[12pt]{exam}
\usepackage{lipsum}

% 支持中文的设置
\usepackage{xeCJK}
\usepackage{fontspec}
\setCJKmainfont[ItalicFont=思源宋体,BoldFont=SourceHanSerifSC-Bold]{Source Han Serif SC}
\newcommand{\KaiTi}{\CJKfontspec{楷体}}%用命令\fzkaiti调用方正楷体简体

\begin{document}
	\large
	\linespread{1.6} \selectfont
	
	\begin{center}
		\textbf{\large 技术面试题}
	\end{center}
	
	\vspace{\baselinestretch}
	
	\begin{questions}
	
	\question
	\textbf{虚函数:}
	\begin{parts}
		\part 你平时是如何使用虚函数的?
		\part 使用虚函数需要注意点什么?
	\end{parts}
	
	\question
	\textbf{内存泄露:}
	\begin{parts}
		\part 内存泄露的处理方式有哪些?
		\part 如何观察和解决内存泄露?
	\end{parts}
	
	
	
	\question
	\textbf{野指针:}
	\begin{parts}
		\part 什么是野指针?
		\part 一般什么时候会出现野指针?
		\part 如何避免出现野指针?
	\end{parts}
	
	\question
	\textbf{多线程:}
	\begin{parts}
		\part 什么是线程安全?列举几种保证线程安全的策略。
		\part 解释一下互斥锁和条件变量在多线程编程中的作用。
	\end{parts}
	
	\question
	\textbf{Qt相关:}
	\begin{parts}
		\part 什么是 Qt 的信号和槽机制?它在多线程环境下有哪些注意事项?
		\part 如何在 Qt 中进行跨平台的编译和部署?
	\end{parts}
	\end{questions}

\end{document}