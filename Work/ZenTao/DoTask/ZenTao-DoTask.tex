% !TeX program = xelatex 

\PassOptionsToPackage{prologue, dvipsnames}{xcolor}
\documentclass[AutoFakeBold,AutoFakeSlant]{beamer}
\usetheme{metropolis}           % Use metropolis theme
\setbeamercovered{transparent}
\usepackage{listings}
\usepackage{grid-system}
\usepackage{ThinctPPT}
\usepackage[font=normalsize,labelfont=sf,textfont=sf]{subcaption} % Use only subcaption, not subfig

% 支持中文的设置
\usepackage{xeCJK}
\usepackage{fontspec}
\setCJKmainfont[ItalicFont=思源宋体,BoldFont=SourceHanSerifSC-Bold]{Source Han Serif SC}
\newcommand{\KaiTi}{\CJKfontspec{楷体}}%用命令\fzkaiti调用方正楷体简体

% other packages
\usepackage{latexsym,amsmath,xcolor,multicol,booktabs,calligra}
\usepackage{graphicx,pstricks,listings,stackengine}
\usepackage{wrapfig}
\usepackage[english]{babel}
\usepackage[font=normalsize,labelfont=sf,textfont=sf]{subcaption}

\title{\textbf{禅道}\\任务执行}
\date{\today}
\author{\includegraphics[width=0.26\linewidth]{LaserMaker}\\软件部~/~王升平}
\begin{document}
	\maketitle
	
	\section{执行任务}
	\subsection{每日工作之前先建任务}
	
	\begin{frame}[fragile]
		\LogoFrametitle{怎样建任务}
		\begin{enumerate}
			\item 修改BUG。建任务,可以与BUG同名。
			\item 写需求。建任务,可以与需求同名。
			\item 写文档。建任务,与具体事务同名。
			\item 测试。具体事务相关。白盒测试需任务细化。
		\end{enumerate}
	\end{frame}
	
	\subsection{任务必须遵循的要点}
	\begin{frame}[fragile]
		\LogoFrametitle{必须要做到的:}
		\begin{center}
			禅道的执行状态,必须要做到每天有{\large \textbf{"执行中"}}
		\end{center} 
	\end{frame}
	
	\begin{frame}[fragile]
		\LogoFrametitle{时间计划}
		\begin{figure}
			\large
			\begin{flushleft}
				1. 开始的估算总工时按照经验填写
			\end{flushleft}
			
			\vspace{1cm}
			
			\begin{minipage}[l]{\linewidth}
				2. 消耗工时按照实际方式填写
				\footnotesize
				\begin{enumerate}
					\item 严重超过时间估算需要与魁哥或者泽贤沟通或者指派给王工
					\item 有明确截止日期的
					\begin{enumerate}
						\item 截止日期前两天(至少提前一天)需反馈
						\item 连续两天没有连续进展需反馈
					\end{enumerate}	
				\end{enumerate}
			\end{minipage}
			
			\vspace{0.5cm}
			
			\begin{flushleft}
				3. 理论上,一天消耗的工时不能小于4小时
			\end{flushleft}
		\end{figure}
	\end{frame}
	
	
	\begin{frame}[fragile]
		\LogoFrametitle{反馈}
		\begin{figure}
			\large
			\begin{minipage}[l]{\linewidth}
				1. 提前做好问题的矛盾点描述
				\footnotesize
				\begin{enumerate}
					\item 问题简单化
					\item 描述有条理
					\item 如果能提供简单Demo复现问题更好
				\end{enumerate}
			\end{minipage}
			
			\vspace{0.5cm}
			
			\begin{minipage}[l]{\linewidth}
				2. 描述解决问题的思路
				\footnotesize
				\begin{enumerate}
					\item 已经进行过的思路探索 123
					\item 将要尝试的思路 123
				\end{enumerate}
			\end{minipage}
		\end{figure}
	\end{frame}
	
	\subsection{其他}
	\begin{frame}[fragile]
		\LogoFrametitle{其他}
		\begin{center}
			后续考虑取消腾讯文档的日志编写,\\改用禅道作为唯一工作日志渠道。
		\end{center} 
	\end{frame}
\end{document}
